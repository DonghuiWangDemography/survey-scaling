
\documentclass{article}
\usepackage[utf8]{inputenc}
\usepackage{textgreek}
\usepackage{graphicx}
\usepackage{float}
\usepackage{booktabs}
\usepackage{booktabs,caption}

\usepackage{xcolor}
\usepackage{sectsty}
\definecolor{purdw}{rgb}{0.67,0.15,0.31}
\sectionfont{\color{purdw}}  % sets colour of sections

%\usepackage{floatrow}
\usepackage[a4paper, total={6in, 8in}]{geometry}

\usepackage{hyperref}
\hypersetup{
    colorlinks=true,
    linkcolor=blue,
    filecolor=magenta,      
    urlcolor=cyan,
}



\title{Aggregate IRT to Estimate Public Opinion  \\
(McGann 2014, 2019)}
\date{\today}
%\author{Donghui Wang}

\begin{document}

\maketitle
\section{Binary scale (McGann 2014)}
\subsection{Model set up}
\begin{enumerate}
    \item Denote individual i's latent attitude at year y is a random variable  $x_{iy}$ that follows normal distribution with mean $\mu_y$ and standard deviation $\sigma_y$. 
    
    \item Denote number of questions being asked is  q where $q \in (1, Q)$  \\
     These q questionnaires differentiate each other in two ways: 
    \subitem (1) how difficult the question is to answer $\lambda_q$ (difficult parameter),   
    \subitem (2) how well the question discriminate  $\alpha_q$ (discrimination parameter)
    
    \item There are four sets of parameters we need to estimate 
        \subitem (1) The central tendency of population attitude.
                     Described by  mean $\mu_y$ and standard deviation $\sigm_y$. 
        \subitem (2) The survey attributes. Described by \textlambda\textsubscript{q} and \textalpha\textsubscript{q}

    
        \item Denote the probability of individual i give correct answer to question q in year y is $e_{iyq}$, which can be expressed as a function of question characteristics and individual i's latent trait. We have a individual model: 
        
     \begin{equation}
        e_{iyq} = \Phi(x_iy,\lambda_q,\alpha_q)
    \end{equation}
%     \footnotesize DW: why normal not logistic ? 
    \item Integrated over the entire population we have 
    
    \begin{equation}
        m_{yq} = \int_{-\infty}^{\infty}
        \Phi(x,\lambda_q,\alpha_q)\phi(x,\mu_y, \sigma_y)dx
    \end{equation}
    
    Integrating, we get 
    \begin{equation}
         m_{yq} = \Phi(\mu_y,\lambda_q, \sqrt{\alpha _q^2 + \sigma_y^2  }  )
    \end{equation}

    where $m_{yq}$ is the expected probability of a random respondent giving a correct answer to question q in year y
 \end{enumerate}
   
\subsection{Model Estimation}
    \begin{enumerate}
        \item Model the total number of correct response to a question using beta-binomial distribution 
        
    \end{enumerate}

\section{Ordinal scale (McGann 2019)}

\subsection{Model set up}
    \begin{enumerate}
        \item For the case of ordinal scale, replace difficulty parameter $\lambda_q$ with a set of parameters  $\lambda_{qg}$,  where g denotes the response level.  
        \item The cumulative probability of answering at least at certain level can be modelled in the same way as binary question
    \end{enumerate}


\section{References}

\begin{itemize}
    \item McGann, Anthony J. 2014. “Estimating the Political Center from Aggregate Data: An Item Response Theory Alternative to the Stimson Dyad Ratios Algorithm.” Political Analysis 22 (1): 115–29.  
    \item McGann, Anthony, Sebastian Dellepiane-Avellaneda, and John Bartle. 2019. “Parallel Lines? Policy Mood in a Plurinational Democracy.” Electoral Studies 58 (June 2018): 48–57. 
\end{itemize}


\section{Model the number of correct responses}
\begin{itemize}
    \item  The proportion of correct response to questionnaire j in year y is modeled as 
    
\end{itemize}

\end{document}